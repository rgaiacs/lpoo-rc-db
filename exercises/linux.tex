\documentclass[letterpaper,11pt]{article}
\usepackage[brazil]{babel}
\usepackage[T1]{fontenc}
\usepackage{ae}
\usepackage[utf8]{inputenc}
\usepackage[dvipsnames]{color}
\usepackage{graphicx}
\usepackage{epsfig}
\usepackage{makeidx}
\usepackage{multicol}
\usepackage{amssymb, amsmath, amsfonts}
\bibliographystyle{plain}
\topmargin		-1 cm
\hoffset		0 cm
\voffset		0 cm
\evensidemargin		0 cm
\oddsidemargin		0 cm
\setlength{\textwidth}{16 cm}
\setlength{\textheight}{22 cm}
\usepackage{lpoo}

\title{ LPOO - Rede de Colaboração \\
Linux - Exercícios }
\author{ Abel Soares Siqueira }


\begin{document}
\maketitle

\begin{description}
  \item[0001] Dentro de sua pasta pessoal, crie um arquivo chamado 
    \verb+ex0001.lpoo+ contendo a listagem detalhado do diretório 
    \verb+/usr/bin+, usando apenas um comando. \easy
  \item[0002] Usando apenas um comando, liste todos os arquivos da pasta
    raiz, separe apenas os que tenham a letra \verb+b+, e substitua
    a palavra \verb+in+ por \verb+omba+. \hard
\end{description}

\end{document}
