\documentclass[a4paper,12pt]{article}
\usepackage[utf8]{inputenc}
\usepackage[T1]{fontenc}
\usepackage{ae}
\usepackage[top=3cm, bottom=3cm, left=2cm, right=2cm]{geometry}
\usepackage[brazil]{babel}
\usepackage{indentfirst}
\usepackage[dvipsnames]{color}
\usepackage{graphicx}
\usepackage{amsmath}
\usepackage{amsfonts}
\usepackage{amssymb}
\usepackage{lpoo}
\bibliographystyle{plain}

\title{ LPOO - Rede de Colaboração \\
ASSUNTO - ID XXX - Exercícios }
\author{ AUTOR }
\date{ Última atualização: \today }

\begin{document}
\maketitle

\begin{abstract}
Este é o padrão de exercícios a ser seguido pela Rede de Colaboração do LPOO.
\end{abstract}

\section{Explicação}
Para os exercícios é utilizado o pacote \lstinline+exercise+ pois este oferece várias
opções e possibilita a construção de um banco de dados de questões. A inclusão
padrão do pacote \lstinline+exercise+ é
\begin{lstlisting}
\usepackage{exercise}
\end{lstlisting}
sendo que no arquivo pdf será incluído os exercícios e as respostas. Se for
desejado omitir as resposta, utilize
\begin{lstlisting}
\usepackage[noanswer]{exercise}
\end{lstlisting}

Para o tema \lstinline+FOO+ deve existir um arquivo \lstinline+FOO-db.tex+ onde os
exercícios são armazenados. Cada exercício é da forma:
\begin{lstlisting}
\begin{Exercise}[label={XXXX}, difficulty=Y, origin={BAR}]
  Pergunta.
\end{Exercise}
\begin{Answer}[ref={XXXX}]
  Resposta.
\end{Answer}
\end{lstlisting}
onde \lstinline+XXXX+ é um número correspondendo ao ID do exercício, \lstinline+Y+ é um
número entre 0 e 3 correspondendo a dificuldade (0 corresponde a fácil e 3 a
difícil) e \lstinline+BAR+ é uma tag para o exercício (muito provavelmente um
subtema). Para mais exemplos, veja o arquivo \lstinline+padrao-db.tex+.

A seleção dos exercícios de \lstinline+FOO-db.tex+ é feita com as linhas abaixo:
\begin{lstlisting}
\ExerciseSelect[origin={BAR}]
\input{FOO-db}
\ExerciseStopSelect
\end{lstlisting}
onde será selecionado todos os exercícios em que o campo \lstinline+origin+ é igual a
\lstinline+BAR+. Também é possível selecionar utilizando o campo \lstinline+label+ e
\lstinline+difficulty+.

\section{Exemplo}
\ExerciseSelect[label={easy,medium,hard,veryhard,question,lstlisting}]
\begin{Exercise}[label={easy}, difficulty=0]
  Esta é uma questão fácil, \verb+difficulty=0+.
\end{Exercise}
\begin{Answer}[ref={easy}]
  Resposta de um exercício fácil.
\end{Answer}
\begin{Exercise}[label={medium}, difficulty=1]
  Esta é uma questão média, \verb+difficulty=1+.
\end{Exercise}
\begin{Answer}[ref={medium}]
  Resposta de um exercício médio.
\end{Answer}
\begin{Exercise}[label={hard}, difficulty=2]
  Esta é uma questão difícil, \verb+difficulty=2+.
\end{Exercise}
\begin{Answer}[ref={hard}]
  Resposta de um exercício difícil.
\end{Answer}
\begin{Exercise}[label={veryhard}, difficulty=3]
  Esta é uma questão muito difícil, \verb+difficulty=3+.
\end{Exercise}
\begin{Answer}[ref={veryhard}]
  Resposta de um exercício muito difícil.
\end{Answer}
\begin{Exercise}[label={lstlisting}, difficulty=0]
  Esta é uma questão que mostra o uso do pacote \verb+lstlisting+.
  \begin{lstlisting}
#include <stdlib.h>
#include <stdio.h>

int main() {
  return 0;
}
  \end{lstlisting}
\end{Exercise}
\begin{Answer}[ref={lstlisting}]
  Resposta de um exercício.
\end{Answer}

\ExerciseStopSelect
\end{document}
