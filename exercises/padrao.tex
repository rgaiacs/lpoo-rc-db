\documentclass[a4paper,12pt]{article}
\usepackage[utf8]{inputenc}
\usepackage[brazil]{babel}
\usepackage[T1]{fontenc}
\usepackage{ae}
\usepackage[top=3cm, bottom=3cm, left=2cm, right=2cm]{geometry}
\usepackage[dvipsnames]{color}
\usepackage{graphicx}
\usepackage{amsmath}
\usepackage{amsfonts}
\usepackage{amssymb}
\usepackage{lpoo}
\usepackage{listings}
\bibliographystyle{plain}

\title{ LPOO - Rede de Colaboração \\
ASSUNTO - ID XXX - Exercícios }
\author{ AUTOR }
\date{ Última atualização: \today }

\begin{document}
\maketitle

Este é o padrão de exercícios a ser seguido.

Para os exercícios é utilizado o pacote \verb+exercise+ pois este oferece várias
opções e possibilita a construção de um banco de dados de questões.

Para o tema \verb+FOO+ deve existir um arquivo \verb+FOO-db.tex+ onde os
exercícios são armazenados. Cada exercício é da forma:
\begin{lstlisting}
\begin{Exercise}[label={XXXX}, difficulty=Y]
  Pergunta.
\end{Exercise}
\begin{Answer}[ref={XXXX}]
  Resposta.
\end{Answer}
\end{lstlisting}
onde \verb+XXXX+ é um número correspondendo ao ID do exercício e \verb+Y+ é um
número entre 0 e 3 correspondendo a dificuldade. Para mais exemplos, veja o
arquivo \verb+padrao-db.tex+.

\begin{Exercise}[label={easy}, difficulty=0]
  Esta é uma questão fácil, \verb+difficulty=0+.
\end{Exercise}
\begin{Answer}[ref={easy}]
  Resposta de um exercício fácil.
\end{Answer}
\begin{Exercise}[label={medium}, difficulty=1]
  Esta é uma questão média, \verb+difficulty=1+.
\end{Exercise}
\begin{Answer}[ref={medium}]
  Resposta de um exercício médio.
\end{Answer}
\begin{Exercise}[label={hard}, difficulty=2]
  Esta é uma questão difícil, \verb+difficulty=2+.
\end{Exercise}
\begin{Answer}[ref={hard}]
  Resposta de um exercício difícil.
\end{Answer}
\begin{Exercise}[label={veryhard}, difficulty=3]
  Esta é uma questão muito difícil, \verb+difficulty=3+.
\end{Exercise}
\begin{Answer}[ref={veryhard}]
  Resposta de um exercício muito difícil.
\end{Answer}
\begin{Exercise}[label={lstlisting}, difficulty=0]
  Esta é uma questão que mostra o uso do pacote \verb+lstlisting+.
  \begin{lstlisting}
#include <stdlib.h>
#include <stdio.h>

int main() {
  return 0;
}
  \end{lstlisting}
\end{Exercise}
\begin{Answer}[ref={lstlisting}]
  Resposta de um exercício.
\end{Answer}

\ExerciseSelect[label={easy0, easy1, medium, hard, veryhard, lstlisting}]
\end{document}
