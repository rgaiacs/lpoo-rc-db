\begin{Exercise}[label={easy}, difficulty=0]
  Esta é uma questão fácil, \lstinline+difficulty=0+.
\end{Exercise}
\begin{Answer}[ref={easy}]
  Resposta de um exercício fácil.
\end{Answer}

\begin{Exercise}[label={medium}, difficulty=1]
  Esta é uma questão média, \lstinline+difficulty=1+.
\end{Exercise}
\begin{Answer}[ref={medium}]
  Resposta de um exercício médio.
\end{Answer}

\begin{Exercise}[label={hard}, difficulty=2]
  Esta é uma questão difícil, \lstinline+difficulty=2+.
\end{Exercise}
\begin{Answer}[ref={hard}]
  Resposta de um exercício difícil.
\end{Answer}

\begin{Exercise}[label={veryhard}, difficulty=3]
  Esta é uma questão muito difícil, \lstinline+difficulty=3+.
\end{Exercise}
\begin{Answer}[ref={veryhard}]
  Resposta de um exercício muito difícil.
\end{Answer}

\begin{Exercise}[label={question}, difficulty=0]
  Se a questão tiver itens, deve-se utilizar o comando \lstinline+\Question+.

  \Question Primeiro item.
  \Question Segundo item.
\end{Exercise}
\begin{Answer}[ref={question}]
  Também deve-se utilizar o camando \lstinline+\Question+ para indicar a
  resposta dos itens.
  \Question Resposta do primeiro item.
  \Question Resposta do segundo item.
\end{Answer}

\begin{Exercise}[label={lstlisting}, difficulty=0]
  Esta é uma questão que mostra o uso do pacote \lstinline+lstlisting+.
  \lstinputlisting{aux/padrao.c}
  Códigos e arquivos que o aluno passa precisar devem ser
  ``impressos''/incluidos no exercício utilizando o comando
  \lstinline+lstinputlisting+ do pacote \lstinline+lstlisting+ pois o \lstinline+Makefile+ gera
  automaticamente um arquivo \lstinline+tar+ contendo estes arquivos.
\end{Exercise}
\begin{Answer}[ref={lstlisting}]
  Resposta de um exercício.
\end{Answer}
